\begin{frame}
  \frametitle{Argument Passing in Java}
  \code[language=java,font=\small]{pass-int.java}
  \begin{itemize}
    \item Primitive types are passed by value
    \item All of \texttt{x}'s bits are copied to the stack
    \item \texttt{foo} is unable to modify \texttt{x}'s value
  \end{itemize}
\end{frame}

\begin{frame}
  \frametitle{Argument Passing in Java}
  \code[language=java,font=\small]{pass-int-array.java}
  \begin{itemize}
    \item Objects are passed by reference
    \item \texttt{foo} can modify the array
  \end{itemize}
\end{frame}

\begin{frame}
  \frametitle{Why This Inconsistency?}
  \begin{itemize}
    \item Primitives: by value
    \item Objects: by reference
    \item Why not everything the same way?
  \end{itemize}
\end{frame}

\begin{frame}
  \frametitle{Reason \#1: Efficiency}
  \begin{itemize}
    \item Say objects are passed by value
    \item In case of array, all elements would have to be copied
    \item Uses a lot of time and memory
    \item References have constant size: 32 or 64 bit
    \item Fit in register
    \item You can give access to arbitrarily large data structures using a single reference
  \end{itemize}
\end{frame}

\begin{frame}
  \frametitle{Reason \#2: Sharing}
  \begin{itemize}
    \item References allow you to ``share'' objects
    \item If objects were always passed by value, you could not pass the \emph{same} object to two different objects
  \end{itemize}

  \begin{center}
    \begin{tikzpicture}[person/.style={fill=white,drop shadow,draw,minimum width=2cm,minimum height=.75cm},scale=.75,transform shape]
      \begin{scope}
        \node[person] (parent1) at (-1.5,2) {parent};
        \node[person] (parent2) at (1.5,2) {parent};
        \node[person] (son) at (-1.5,0) {son};
        \node[person] (daughter) at (1.5,0) {daughter};
        \draw[-latex] (son) -- (parent1);
        \draw[-latex] (daughter) -- (parent2);
      \end{scope}

      \node at (3.5,1) {vs};

      \begin{scope}[xshift=6.5cm]
        \node[person] (parent) at (0,2) {parent};
        \node[person] (son) at (-1.5,0) {son};
        \node[person] (daughter) at (1.5,0) {daughter};
        \draw[-latex] (son) -- (parent);
        \draw[-latex] (daughter) -- (parent);
      \end{scope}
    \end{tikzpicture}
  \end{center}
\end{frame}

\begin{frame}
  \frametitle{Reason \#3: Recursion and \texttt{null}}
  \code[language=java]{person.java}
  \begin{itemize}
    \item Without references, no recursive data structures
    \item \texttt{Person} would be infinitely large
    \item Thanks to references, a \texttt{Person} is only 64/128 bit large
    \item \texttt{null} can be used to end chains
  \end{itemize}
\end{frame}

\begin{frame}
  \frametitle{Reason \#4: Giving Access}
  \code[language=java]{this.java}
  \begin{itemize}
    \item Without references, functions would never be able to modify their arguments
    \item Without references, \texttt{this} would be a copy
    \item Methods would never be able to modify the object's fields
  \end{itemize}
\end{frame}

\begin{frame}
  \frametitle{References in Java}
  \begin{itemize}
    \item Whether \texttt{T x} is a reference, depends on the type \texttt{T}
          \begin{itemize}
            \item If \texttt{T} is primitive, then it's not a reference
            \item If \texttt{T} is a class, then it's a reference
          \end{itemize}
    \item No extra syntax needed to discern references from values
  \end{itemize}
  \begin{center}
    \begin{tabular}{ccc}
        & \textbf{primitive} & \textbf{class} \\
      \toprule
      \textbf{value} & yes & not supported \\
      \textbf{reference} & not supported & yes \\
    \end{tabular}
  \end{center}
\end{frame}

\begin{frame}
  \frametitle{Pointers in \cpp}
  \begin{center}
    \begin{tabular}{ccc}
        & \textbf{primitive} & \textbf{class} \\
      \toprule
      \textbf{value} & yes & yes \\
      \textbf{pointer} & yes & yes \\
    \end{tabular}
  \end{center}
  \begin{itemize}
    \item \cpp\ provides \emph{pointers}
    \item Fulfill same role as references as in Java
    \item \cpp\ do not treat primitives and classes differently
          \begin{itemize}
            \item Java does not allow reference to primitive
            \item \cpp\ allows pointer to primitive
          \end{itemize}
    \item How to know if you want pointer or not?
    \item Extra syntax needed!
    \item Value: \texttt{T x}
    \item Pointer: \texttt{T* x}
  \end{itemize}
\end{frame}

%%% Local Variables:
%%% mode: latex
%%% TeX-master: "pointers"
%%% End:
