\section{Static Allocation}

\begin{frame}
  \tableofcontents[currentsection]
\end{frame}

\begin{frame}
  \frametitle{Static Allocation}
  \begin{itemize}
    \item Simplest allocation mechanism
    \item Fastest
    \item Most restrictive
    \item Was only available allocation method in first programming language (Fortran, early 1950s)
  \end{itemize}
  \vskip1cm
  \begin{center}
    {\color{red}\Huge\danger}
    \parbox{8cm}{\centering 
      {\color{red}\Huge \textsc{warning}} \\
      Examples that follow are what-ifs \\
      They do not show how really \cpp\ works
    }
    {\color{red}\Huge\danger}
  \end{center}
\end{frame}

\begin{frame}
  \frametitle{How Does It Work?}
  \notrealcpp
  \begin{itemize}
    \item Compiler scans source code for variables
          \begin{itemize}
            \item Parameters, locals, globals, \dots
          \end{itemize}
    \item The compiler assigns a fixed address to each variable
  \end{itemize}
  \code[language=c++14,frame=none]{allocation.cpp}
  \begin{tikzpicture}[overlay,remember picture,allocation/.style={-latex,thick}]
    \coordinate (upperleft) at (6,4);

    \begin{scope}[scale=.5]
      \coordinate (double) at (upperleft);
      \coordinate (int) at ($ (upperleft) + (0,-1) $);
      \coordinate (char) at ($ (upperleft) + (4,-1) $);

      \visible<2->{
        \draw[fill=red!50] (double) rectangle ++(8,-1);
      }

      \visible<3->{
        \draw[fill=green!50] (int) rectangle ++(4,-1);
      }

      \visible<4->{
        \draw[fill=blue!50] (char) rectangle ++(1,-1);
      }

      \draw (upperleft) grid ++(8,-4);
    \end{scope}

    \visible<2>{
      \draw[allocation] (global) -- (double);
    }

    \visible<3>{
      \draw[allocation] (param) -- ($ (int) + (0,-0.25) $);
    }

    \visible<4>{
      \draw[allocation] (local) -- ($ (char) + (0,-0.25) $);
    }
  \end{tikzpicture}
\end{frame}

\begin{frame}
  \frametitle{Consequences of Static Allocation}
  \notrealcpp
  \begin{itemize}
    \item Compiler needs to have full knowledge at compile-time
  \end{itemize}
  \code{array-creation.cpp}
\end{frame}

\begin{frame}
  \frametitle{Consequences of Static Allocation}
  \notrealcpp
  \begin{itemize}
    \item Local variables ``remember'' their values
  \end{itemize}
  \code[language=c++14]{static-locals.cpp}
\end{frame}

% \begin{frame}
%   \frametitle{Quick Intermezzo}
%   \begin{itemize}
%     \item You can actually make this work in \cpp
%     \item Static local variables remember their values
%   \end{itemize}
%   \code{static-locals-working.cpp}
% \end{frame}

\begin{frame}
  \frametitle{Consequences of Static Allocation}
  \notrealcpp
  \begin{itemize}
    \item Recursion does not work
  \end{itemize}
  \code[language=c++14]{static-recursion.cpp}
\end{frame}

\begin{frame}
  \frametitle{Static Allocation}
  \begin{procontralist}
    \pro Very fast
    \pro No bookkeeping needed at runtime
    \pro No out-of-memory errors possible at runtime
    \pro Entire memory footprint known beforehand
    \con Data structure size must be known at compile time
    \con Full recompilation needed to allow larger data structures
    \con Cannot make use of extra RAM:
         If it's been compiled to work with 1MB but you have 8GB, too bad
    \con No recursion possible
  \end{procontralist}
\end{frame}

%%% Local Variables:
%%% mode: latex
%%% TeX-master: "allocation-methods"
%%% End:
