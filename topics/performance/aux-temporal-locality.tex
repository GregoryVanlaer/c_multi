\subsection{Temporal Locality}

\frame{\tableofcontents[currentsubsection]}

\begin{frame}
  \frametitle{Temporal Locality}
  \begin{itemize}
    \item Assumption: code often accesses same memory locations repeatedly in short time
    \item Opposite of: code accesses some memory location once, and then leaves it alone for a long time
    \item This idea is known as \emph{temporal locality}
    \item How to make use of temporal locality?
  \end{itemize}
\end{frame}

\begin{frame}
  \frametitle{Cache (Simplified Explanation)}
  \begin{center}
    \begin{tikzpicture}[part/.style={draw,minimum width=1.5cm,minimum height=.5cm,font=\small\scshape},
                        cpu/.style={part,fill=blue!50},
                        cache/.style={part,fill=green!50},
                        ram/.style={part,fill=red!50},
                        fetch/.style={-latex,thick},
                        slow fetch/.style={fetch,red},
                        fast fetch/.style={fetch,green!50!black},
                        caption/.style={font=\small\scshape}]
      \pgfdeclarelayer{background}
      \pgfdeclarelayer{foreground}
      \pgfsetlayers{background,main,foreground}

      \node[cpu] (cpu) at (0,0) {cpu};

      \node[cache] (cache line 1) at (2.5,0.9) {\only<5-10>{a}\only<11-19>{e}\only<20->{d}};
      \node[cache] (cache line 2) at (2.5,0.3) {\only<8-13>{b}\only<14->{a}};
      \node[cache] (cache line 3) at (2.5,-0.3) {\only<8-15>{c}\only<16->{b}};
      \node[cache] (cache line 4) at (2.5,-0.9) {\only<8-17>{d}\only<18->{c}};

      \node[ram] (ram 1) at (8,1.5) {a};
      \node[ram] (ram 2) at (8,0.9) {b};
      \node[ram] (ram 3) at (8,0.3) {c};
      \node[ram] (ram 4) at (8,-0.3) {d};
      \node[ram] (ram 5) at (8,-0.9) {e};
      \node[ram] (ram 6) at (8,-1.5) {f};
      \node[anchor=south,inner sep=0mm,yshift=5pt] (ram above) at (ram 1.north) {\vdots};
      \node[anchor=north,inner sep=0mm] (ram below) at (ram 6.south) {\vdots};

      \begin{pgfonlayer}{background}
        \draw[fill=green!25] let \p1=(cache line 1.north west),
                                 \p2=(cache line 4.south east)
                             in                             
                             ($ (\x1,\y1) + (-0.25,0.25) $) coordinate (cache upper left)
                             rectangle
                             ($ (\x2,\y2) + (0.25,-0.25) $) coordinate (cache lower right);

        \path let \p1=(cache upper left),
                  \p2=($ (cache upper left) ! 0.5 ! (cache lower right) $),
                  \p3=(cache lower right)
              in
              coordinate (cache center) at (\p2)
              coordinate (cache left) at (\x1,\y2)
              coordinate (cache right) at (\x3,\y2);

        \draw[fill=red!25] let \p1=(ram 1.north west),
                               \p2=(ram 1.south east),
                               \p3=(ram above.north west),
                               \p4=(ram below.south east)
                           in
                           ($ (\x1,\y3) + (-0.25,0.25) $)
                           rectangle
                           ($ (\x2,\y4) + (0.25,-0.25) $);

        \path let \p1=($ (cache line 1.north west) ! 0.5 ! (cache line 4.south east) $),
                  \p2=(cache line 4)
              in
              node[anchor=north,caption] at (\x1,\y2-0.5cm) {cache};
        \path let \p1=($ (ram 1.north west) ! 0.5 ! (ram 6.south east) $),
                  \p2=(ram below.south east)
              in
              node[anchor=north,caption] at (\x1,\y2-0.25cm) {ram};
      \end{pgfonlayer}

      \visible<3>{
        \draw[fast fetch] (cpu) -- (cache left) node[above,midway,font=\tiny] {read A};
      }

      \visible<4>{
        \draw[slow fetch] (cache right) -- ++(0.5,0) |- (ram 1);
      }

      \visible<5>{
        \draw[slow fetch] (ram 1.west) -- ++(-0.75,0) |- (cache line 1.east);
        \draw[fast fetch] (cache line 1.west) -- ++(-0.5,0) |- (cpu);
      }

      \visible<6>{
        \draw[fast fetch] (cpu) -- (cache left) node[above,midway,font=\tiny] {read A};
      }

      \visible<7>{
        \draw[fast fetch] (cache line 1.west) -- ++(-0.5,0) |- (cpu);
      }

      \visible<10>{
        \draw[fast fetch] (cpu) -- (cache left) node[above,midway,font=\tiny] {read E};
      }

      \visible<11>{
        \draw[slow fetch] (ram 5.west) -- ++(-0.75,0) |- (cache line 1.east);
        \draw[fast fetch] (cache line 1.west) -- ++(-0.5,0) |- (cpu);
      }

      \visible<13>{
        \draw[fast fetch] (cpu) -- (cache left) node[above,midway,font=\tiny] {read A};
      }

      \visible<14>{
        \draw[slow fetch] (ram 1.west) -- ++(-0.75,0) |- (cache line 2.east);
        \draw[fast fetch] (cache line 2.west) -- ++(-0.5,0) |- (cpu);
      }

      \visible<15>{
        \draw[fast fetch] (cpu) -- (cache left) node[above,midway,font=\tiny] {read B};
      }

      \visible<16>{
        \draw[slow fetch] (ram 2.west) -- ++(-0.75,0) |- (cache line 3.east);
        \draw[fast fetch] (cache line 3.west) -- ++(-0.5,0) |- (cpu);
      }

      \visible<17>{
        \draw[fast fetch] (cpu) -- (cache left) node[above,midway,font=\tiny] {read C};
      }

      \visible<18>{
        \draw[slow fetch] (ram 3.west) -- ++(-0.75,0) |- (cache line 4.east);
        \draw[fast fetch] (cache line 4.west) -- ++(-0.5,0) |- (cpu);
      }

      \visible<19>{
        \draw[fast fetch] (cpu) -- (cache left) node[above,midway,font=\tiny] {read D};
      }

      \visible<20>{
        \draw[slow fetch] (ram 4.west) -- ++(-0.75,0) |- (cache line 1.east);
        \draw[fast fetch] (cache line 1.west) -- ++(-0.5,0) |- (cpu);
      }
    \end{tikzpicture}
  \end{center}
  \begin{overprint}
    \onslide<2>
    \begin{center}
      Say cache can contain 4 lines
    \end{center}

    \onslide<3>
    \begin{center}
      CPU wants to read RAM location A
    \end{center}

    \onslide<4>
    \begin{center}
      Cache is empty at this point \\ We need to wait for RAM to answer
    \end{center}

    \onslide<5>
    \begin{center}
      After a few cycles, RAM answers \\
      Result is copied to cache
    \end{center}

    \onslide<6>
    \begin{center}
      CPU wants to read A again \\
    \end{center}

    \onslide<7>
    \begin{center}
      A is already in cache; CPU receives answer quickly \\
      All subsequent reads to A will be fast
    \end{center}

    \onslide<8>
    \begin{center}
      Say CPU reads B, C, D, then cache will look like this
    \end{center}

    \onslide<9>
    \begin{center}
      If CPU keeps working on A, B, C, D, everything will go fast
    \end{center}

    \onslide<10>
    \begin{center}
      What if the CPU now wants to read E?
    \end{center}

    \onslide<11>
    \begin{center}
      Cache is full so one of its lines needs to be recycled \\
      Let's say the oldest one (A) is sacrificed
    \end{center}

    \onslide<12>
    \begin{center}
      As long as CPU restricts itself to B, C, D, E, \\
      everything will be fine
    \end{center}

    \onslide<13-20>
    \begin{center}
      What if the CPU now starts reading A, B, C, D, E cyclically?
    \end{center}

    \onslide<21>
    \begin{center}
      This is disastrous as all reads will be slow
    \end{center}
  \end{overprint}
\end{frame}

\begin{frame}
  \frametitle{Cache}
  \begin{itemize}
    \item For cache to work well, it is important that CPU keeps accessing the same memory locations
    \item If CPU jumps around too much, cache lines will be recycled each time and performance
          will be as bad as if there is no cache
  \end{itemize}
\end{frame}

\begin{frame}
  \frametitle{Benchmark}
  \structure{Setup}
  \begin{itemize}
    \item We start with large block of memory
    \item We split this memory up in $N$ blocks
    \item For each block in turn, we access its contents $k$ times
  \end{itemize}
  \vskip5mm
  \structure{Expectations}
  \begin{itemize}
    \item First block access will be slowest (not yet in cache)
    \item Subsequent access should be fast, if block fits in cache
    \item If block too large, even subsequent access should be slow
  \end{itemize}
\end{frame}

\begin{frame}
  \frametitle{Benchmark}
  \begin{center}
    \begin{tikzpicture}
      \tikzmath{
        real \xdiv;
        real \ydiv;
        \xdiv=2;
        \ydiv=600;
      }
      \draw[thin,gray] (0,0) grid (10,6);
      \foreach[evaluate={\i/\xdiv} as \x,
               evaluate={\j/\ydiv} as \y,
               remember=\x as \lastx (initially 0),
               remember=\y as \lasty (initially 0)] \i/\j in 
               {1/796,
                2/1001,
                3/1016,
                4/892,
                5/832,
                6/808,
                7/806,
                8/890,
                9/823,
                10/809,
                11/831,
                12/895,
                13/870,
                14/872,
                15/959,
                16/1033,
                17/1829,
                18/2906,
                19/3308} {
        \ifnum\i>1
          \draw[red] (\lastx,\lasty) -- (\x,\y);
        \fi
      }

      \foreach[evaluate={\i/\xdiv} as \x,
               evaluate={\j/\ydiv} as \y,
               remember=\x as \lastx (initially 0),
               remember=\y as \lasty (initially 0)] \i/\j in 
               {1/744,
                2/848,
                3/865,
                4/791,
                5/755,
                6/749,
                7/751,
                8/769,
                9/767,
                10/792,
                11/788,
                12/775,
                13/790,
                14/824,
                15/816,
                16/997,
                17/1732,
                18/2665,
                19/3084} {
        \ifnum\i>1
          \draw[green] (\lastx,\lasty) -- (\x,\y);
        \fi
      }

      \foreach[evaluate={\i/\xdiv} as \x,
               evaluate={\j/\ydiv} as \y,
               remember=\x as \lastx (initially 0),
               remember=\y as \lasty (initially 0)] \i/\j in 
               {1/729,
                2/793,
                3/783,
                4/771,
                5/770,
                6/731,
                7/758,
                8/736,
                9/750,
                10/780,
                11/737,
                12/748,
                13/767,
                14/786,
                15/789,
                16/807,
                17/1460,
                18/2620,
                19/3046} {
        \ifnum\i>1
          \draw[blue] (\lastx,\lasty) -- (\x,\y);
        \fi
      }

      \foreach[evaluate={\i/\xdiv} as \x,
      evaluate={\j/\ydiv} as \y,
               remember=\x as \lastx (initially 0),
               remember=\y as \lasty (initially 0)] \i/\j in 
               {1/728,
                2/754,
                3/745,
                4/728,
                5/721,
                6/717,
                7/718,
                8/721,
                9/723,
                10/733,
                11/729,
                12/726,
                13/742,
                14/769,
                15/784,
                16/802,
                17/1449,
                18/2616,
                19/3120} {
        \ifnum\i>1
          \draw[purple] (\lastx,\lasty) -- (\x,\y);
        \fi
      }
    \end{tikzpicture}
  \end{center}
\end{frame}
