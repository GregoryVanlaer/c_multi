\subsection{Spatial Locality}

\frame{\tableofcontents[currentsubsection]}

\begin{frame}
  \frametitle{Spatial Locality}
  \begin{itemize}
    \item Other typical usage pattern
    \item CPU often access memory range sequentially
          \begin{itemize}
            \item E.g.~accessing all elements in an array one after the other
          \end{itemize}
    \item How to make use of this?
  \end{itemize}
\end{frame}

\begin{frame}
  \frametitle{Prefetching}
  \begin{itemize}
    \item When CPU starts accessing memory locations \texttt{p}, \texttt{p+1}, \texttt{p+2},
          already ask RAM for \texttt{p+10}
    \item By the time CPU is done processing \texttt{p}\dots\texttt{p+9}, the rest of the
          data has already been fetched from RAM
  \end{itemize}
\end{frame}

\begin{frame}
  \frametitle{Benchmark}
  \begin{itemize}
    \item Accessing array sequentially: \SI{77}{\milli\second}
    \item Accessing array at random indices: \SI{1848}{\milli\second}
  \end{itemize}
\end{frame}


%%% Local Variables:
%%% mode: latex
%%% TeX-master: "performance"
%%% End:
